\documentclass[a4paper,12pt]{article}

% Time-stamp: <2014/08/09 18:00:55 geraint>

\usepackage{amsmath,amssymb}
\usepackage{fullpage}
\usepackage[UKenglish]{isodate}
\setlength{\parindent}{0pt}
\setlength{\parskip}{\baselineskip}
\hyphenpenalty=10000
\tolerance=500

\isodate
\title{Laplace example: LRC step}
\author{Geraint Bevan
  {\small
    PhD CEng MIET CSci MInstMC FHEA}
  \\\ \\
  geraint.bevan@gcu.ac.uk
  \\\ \\
  Department of Engineering\\
  Glasgow Caledonian University,
  Glasgow G4 0BA, Scotland}

\begin{document}
\maketitle

\part{Electrical}
An electrical LRC circuit is modelled by transfer function $G(s)$:
\begin{align*}
  G(s) &= \frac{V_c}{V_{0}}(s) = \frac{1}{L C s^2 + R C s + 1}
\end{align*}
where $V_0$ is the applied EMF (volts, V), $V_c$ is the potential difference across the capacitor (volts, V),
$L$ is the inductance (henries, H), $R$ is the resistance (ohms, $\Omega$) and $C$ is the capacitance (farads, F).

For given values of $L$, $R$ and $C$, determine the potential difference across the capacitor $V_c(t)$ as a function of time in response to a unit step input $V_0(t)$.
\begin{align*}
  V_0(t) &= \begin{cases} 0 & \text{if } t \leq 0 \\ 1 & \text{if } t > 0 \end{cases} &
  V_0(s) &= \frac{1}{s}
\end{align*}

\newpage
\section*{Over-damped, $G(s)$ has 2 real roots:\\ $(RC)^2>4LC $}
Example: $L=0.5$~H, $R=6~\Omega$, $C=0.1$~F. % s=-2,-10
\begin{align*}
  V_c(s) &= G(s) \times V_0(s)
  = \frac{1}{0.05 s^2 + 0.6 s + 1} \times \frac{1}{s}
  = \frac{20}{s^2 + 12 s + 20} \times \frac{1}{s}
\end{align*}

\subsection*{Partial fraction expansion}
Factorise $G(s)$, using $s = \frac{-b\pm\sqrt{b^2-4ac}}{2a}$ if necessary, to find the denominators.
\begin{align*}
  s &= \frac{-0.6\pm\sqrt{0.6^2-4\times 0.05}}{2\times 0.05} & s=-2~\text{or }s=-10
\end{align*}
Hence
\begin{align*}
  V_c(s) &=\frac{1}{0.05 s^2 + 0.6 s + 1} \times \frac{1}{s}
  = \frac{20}{s(s+2)(s+10)}
  = \frac{A}{s} + \frac{B}{s+2} + \frac{C}{s+10}
\end{align*}
Using Heaviside's Quick Cover-Up method
\begin{align*}
  A &= 1 & B &= -\frac{5}{4} = -1.25 & C &= \frac{1}{4} = 0.25
\end{align*}
\begin{align*}
  V_c(s) &= \frac{1}{s} - \frac{1.25}{s+2} + \frac{0.25}{s+10} &\implies
  V_c(t) &= 1 - 1.25 e^{-2t} + 0.25 e^{-10t}
\end{align*}

\newpage
\section*{Critically-damped, $G(s)$ has repeated real root:\\$(RC)^2=4LC $}
Example: $L=1$~H, $R=20~\Omega$, $C=0.01$~F. % s=-10
\begin{align*}
  V_c(s) &= G(s) \times V_0(s)
  = \frac{1}{0.01 s^2 + 0.2 s + 1} \times \frac{1}{s}
  = \frac{100}{s^2 + 20 s + 100} \times \frac{1}{s}
\end{align*}
Factorise $G(s)$, using $s = \frac{-b\pm\sqrt{b^2-4ac}}{2a}$ if necessary, to find the denominators.
\begin{align*}
  s &= -10~\text{(repeated)}
\end{align*}
\begin{align*}
  V_c(s) &= \frac{100}{s^2 + 20 s + 100} \times \frac{1}{s}
  = \frac{100}{s(s+10)(s+10)}
  = \frac{A}{s} + \frac{B}{s+10} + \frac{C}{(s+10)^2}
\end{align*}
Cross-multiplying and equating coefficients of $s^n$
\begin{align*}
  100 &= A(s+10)^2 + Bs(s+10) + Cs = (A+B)s^2 + (20A+10B+C)s + 100A
\end{align*}
\begin{align*}
  s^2: && 0 &= A+B  &&\implies B = -A\\
  s^1: && 0 &= 20A+10B+C &&\implies C = -10A\\
  s^0: && 100 &= 100 A && \implies A = 1
\end{align*}
Hence
\begin{align*}
  A &= 1 & B &= -1 & C &= -10
\end{align*}
\begin{align*}
  V_c(s) &= \frac{1}{s} - \frac{1}{s+10} - \frac{10}{(s+10)^2} &\implies
  V_c(t) &= 1 - e^{-10t} -10te^{-10t}
\end{align*}
\newpage
\section*{Under-damped, $G(s)$ has complex pair of roots:\\$(RC)^2<4LC $}
Example: $L=1$~H, $R=6~\Omega$, $C=0.1$~F. % s=-3+/-j
\begin{align*}
  V_c(s) &= G(s) \times V_0(s)
  = \frac{1}{0.1 s^2 + 0.6 s + 1} \times \frac{1}{s}
  = \frac{10}{s^2 + 6 s + 10} \times \frac{1}{s}
\end{align*}
Factorising would yield complex roots, so it is better to keep the quadratic whole when generating partial fractions.
\begin{align*}
  V_c(s) &= \frac{10}{s^2+6 s + 10} \times \frac{1}{s} = \frac{A s + B}{s^2 + 6 s + 10} + \frac{C}{s}
\end{align*}
Cross-multiplying and equating coefficients of $s^n$
\begin{align*}
  10 &= (As + B)s + C(s^2+6s+10) = (A+C) s^2 + (B+6C) s + 10 C
\end{align*}
\begin{align*}
  s^2: && 0 &= A+C  &&\implies A = -C\\
  s^1: && 0 &= B+6C &&\implies B = -6C\\
  s^0: && 10 &= 10 C && \implies C = 1
\end{align*}
Hence
\begin{align*}
  A &= -1 & B &= -6 & C &= 1
\end{align*}
\begin{align*}
  V_c(s) &= \frac{1}{s} - \frac{s+6}{s^2+6s+10}
\end{align*}


The inverse transform of the quadratic term can be found by
``completing the square'' and using the frequency shift theorem
($F(s+a)=e^{-at}F(s)$).

After studying a standard table of transforms,o which should
include the transforms $\frac{\omega}{s^2+\omega^2}$ and
$\frac{s}{s^2+\omega^2}$, we wish to express the quadratic in the
denominator ($s^2+6s+10$) in the form $(s+a)^2+\omega^2$. To
complete the square, we choose $a=6/2=3$, i.e. half the
coefficient of $s^1$ and then solve for $\omega$.
\begin{align*}
  s^2+6s+10 &= \left(s+\frac{6}{2}\right)^2 + \omega^2 \\
  &= s^2 + 6s + 9 + \omega^2 
  &\implies \omega^2 = 1 \\
  \therefore G(s) = \frac{10}{s^2+6s+10} &= 10 \times \frac{1}{(s+3)^2+1^2}
\end{align*}

Returning then to $V_c(s)$,
\begin{align*}
  V_c(s) &= G(s)\times V_0(s) = \frac{1}{s} - \frac{s+6}{(s+3)^2+1^2}
  & = \frac{1}{s} - \frac{s+3}{(s+3)^2+1^2} - 3\times \frac{1}{(s+3)^2+1^2} \\
  V_c(t) &= 1 - e^{-3t} \cos t - 3 e^{-3t} \sin t
\end{align*}

\newpage

\part{Mechanical}
A mechanical mass-damper-spring system is modelled by the transfer function $G(s)$:
\begin{align*}
  G(s) &= \frac{X}{F}(s) = \frac{1}{ms^2 + f_v s + K}
\end{align*}
where $F$ is the applied force (newtons, N), $X$ is the displacement (metres, m),
$m$ is the mass (kilograms, kg), $f_v$ is the damping force (newton seconds per metre, N$\cdot$s/m) and $K$ is the spring stiffness (newtons per metre, N/m).

For given values of $m$, $f_v$, $K$, determine the displacement $X(s)$ as a function of time in response to a unit step input $F(t)$.
\begin{align*}
  F(t) &= \begin{cases} 0 & \text{if } t \leq 0 \\ 1 & \text{if } t > 0 \end{cases} &
  F(s) &= \frac{1}{s}
\end{align*}

\section*{Over-damped, $G(S)$ has 2 real roots:\\ $f_v^2>4mK$}
Example: $m=1$~kg, $f_v=3$~N$\cdot$s/m, $K=2$~N/m. % s=-1,-2
\section*{Critically-damped, $G(s)$ has repeated real root:\\$f_v^2=4mK$}
Example: $m=1$~kg, $f_v=2$~N$\cdot$s/m, $K=1$~N/m. % s=-1
\section*{Under-damped, $G(s)$ has complex pair of roots:\\$f_v^2<4mK$}
Example: $m=1$~kg, $f_v=2$~N$\cdot$s/m, $K=2$~N/m. % s=-1+/-j

Numbers differ, but the method for each case is identical to the electrical examples.

\end{document}
